\section{Introduzione al testo}

Il testo che segue � la traduzione, in italiano, delle slide ``Network Monitoring in Practice'' del professor Luca Deri. Le slide originali possono essere scaricate dal link \url{http://luca.ntop.org/Teaching/tm2010.pdf}.

La traduzione non � stata fatta in maniera del tutto fedele: alcune volte sono state fatte delle aggiunte volte a chiarire (si spera) meglio i concetti.

In questo primo capitolo verranno spiegati alcuni argomenti di background che nel testo sono dati per scontato.

\subsection{TODO}
Cosa da inserire in questo capitolo:
\begin{itemize}
\item Pila ISO/OSI
\item TCP handshake
\item Header IP
\item VLAN
\item Cosa fa e a che livello (ISO/OSI) opera:
  \begin{itemize}
  \item hub
  \item bridge
  \item switch
  \item router
  \end{itemize}
\item Due righe su SNMP (architettura agent-manager) e MIB
\item Com'� fatto un cavo di rete (RX/TX, etc\ldots)
\item Come funziona una ethernet (CMSA e CMSA/CD)
\end{itemize}

Cose da inserire nel glossario:
\begin{itemize}
\item Benchmark
\item Frame Relay
\item NAT
\item CDP
\item Failover
\item IPX
\item NeBEUI
\item RTP
\item NetBIOS
\item AppleTalk
\item TCP
\item UDP
\item ICMP
\item MPLS
\item SCTP
\item IDS
\item BPF
\item MRGT
\item SAP
\item SPAM
\item Workstation
\item SMTP
\item ICMP
\item Backbone
\item NIC
\item NPU
\end{itemize}
Una volta fatto il glossario vanno controllate le note a pi� di pagina ed eliminati i termini che sono stati spostati nel glossario.