\documentclass[a4paper,11pt]{article}

% packages %
\usepackage[utf8]{inputenc}              % input unicode
\usepackage[margin=2cm]{geometry}        % impostazione margini
\usepackage[italian]{babel}              % lingua italiana
\usepackage[autosize]{dot2texi}          % graphviz
\usepackage{tikz}                        % graphviz
\usepackage{color}                       % colori
\usepackage[pdfborder={0 0 0}]{hyperref} % collegamenti ipertestuali senza bordo
\usepackage[all]{hypcap}                 % corregge i riferimenti alle figure

% info %
\title{\Huge\tt my-pcap}
\author{Andrea Cardaci}
\date{SGR --- 2009}

% info pdf %
\pdfinfo{/Author(Andrea Cardaci) /Title(my-pcap)}

% font %
\renewcommand*\familydefault{\sfdefault}

% custom %
\definecolor{red}{rgb}{0.7,0,0}
\newcommand\code[1]{\textcolor{red}{\texttt{#1}}}

% document %
\begin{document}

\maketitle

\section{Caratteristiche} \label{par:features}
\code{my-pcap} cattura da una o tutte le interfacce disponibili
fermandosi se interrotto oppure se è stato raggiunto il numero massimo
di pacchetti specificati. Attualmente vengono analizzati solo i
pacchetti \code{TCP} e \code{UDP} su \code{IPv4} (il riconoscimento
viene delegato a un filtro \code{BPF}). Le opzioni di cattura
comprendono inoltre la modalità promiscua e il filtraggio in base alla
direzione dei pacchetti.

\subsection{Sniffer di pacchetti}
La cosa più semplice che è possibile fare è il dump dei pacchetti
intercettati mostrando alcune informazioni fondamentali come la
sorgente e la destinazione.

\subsection{Analizzatore di flussi} \label{par:flows}
\code{my-pcap} può memorizzare i pacchetti catturati compattandoli in
flussi monodirezionali. Un flusso contiene un certo numero di
informazioni relative ai pacchetti che lo compongono, divise in:

\begin{enumerate}
\item chiave:
  \begin{itemize}
  \item IP sorgente;
  \item porta sorgente;
  \item IP destinazione;
  \item porta destinazione.
  \end{itemize}
\item data:
  \begin{itemize}
    \item numero di pacchetti;
    \item dimensione in byte del traffico;
    \item timestamp del primo pacchetto ricevuto;
    \item timestamp dell'ultimo pacchetto ricevuto.
  \end{itemize}
\end{enumerate}

Quando un flusso è attivo/inattivo da troppo tempo viene considerato
scaduto, estratto da \code{my-pcap} e restituito, ad esempio
attraverso una stampa.

\subsection{Statistiche}
Durante la cattura \code{my-pcap} conta il numero di pacchetti che ha
ricevuto facendo distinzione tra \code{TCP} e \code{UDP}. Questi dati
vengono inseriti in un database round-robin (\code{RRDtool}) che ne
tiene traccia per un certo periodo; è possibile disegnare un grafico
dell'andamento: numero di pacchetti al secondo.

\subsection{Interfaccia HTTP}
Opzionalmente \code{my-pcap} può attivare un server HTTP (utilizzando
\code{libmicrohttpd}) attraverso il quale mostrare in una pagina HTML
gli ultimi flussi estratti e il suddetto grafico.

\section{Struttura}
In figura \ref{fig:structure} è mostrata un'approssimazione della
struttura interna di \code{my-pcap}; di seguito verrà illustrato il
funzionamento dei vari thread.

\begin{figure}
  \centering
  \begin{dot2tex}[dot,options=-s]
    digraph
    {
      rankdir = "BT"

      node [ shape = "circle" ]
      capturer [ label = "Capturer" ]
      updater1 [ label = "Updater 1" ]
      updater2 [ label = "Updater 2" ]
      updatern [ label = "Updater n" ]
      flowdumper [ label = "Flow Dumper" ]

      node [ shape = "box" ]
      libpcap [ label = "libpcap" ]
      queue1 [ label = "Queue 1" ]
      queue2 [ label = "Queue 2" ]
      queuen [ label = "Queue n" ]
      hashtable [ label = "Hash Table" ]
      flowoutput [ label = "Flow Output" ]

      libpcap -> capturer

      capturer -> queue1 -> updater1 -> hashtable
      capturer -> queue2 -> updater2 -> hashtable
      capturer -> queuen -> updatern -> hashtable

      hashtable -> flowdumper -> flowoutput
    }
  \end{dot2tex}
  \caption{\small\it Rappresentazione della struttura interna di
    \code{my-pcap}: i cerchi indicano i thread, mentre i rettangoli le
    risorse; le freccie possono essere interpretate come: ``una
    risorsa è letta da un thread'' e ``un thread scrive su una
    risorsa''.}
  \label{fig:structure}
\end{figure}

\subsection{Capturer}
La cattura viene effettuata da un thread tramite la funzione
\code{pcap\_loop}. Il limite di un singolo thread è per evitare attese
inutili su un mutex, che in caso contrario sarebbe
necessario\footnote{a meno di meccanismi come \code{PF\_RING}} a
prevenire race conditions.

Il compito del thread \code{Capturer} è tuttavia limitato (oltre
ovviamente a chiedere il prossimo pacchetto a \code{libpcap}) a
calcolare l'hash del pacchetto (come mostrato in figura
\ref{fig:hash}) e in base a questa inserirlo (non tutto il pacchetto
ma solo le informazioni rilevanti, vedere~§~\ref{par:flows}) nella
coda appropriata.

\begin{figure}
  \centering
  \begin{dot2tex}[pgf]
    digraph
    {
      d2toptions = "--valignmode=dot"
      d2tgraphstyle = "anchor=base"
      hash [ label = "{IP sorgente|IP destinatario|{Porta sorgente|Porta destinatario}}" shape = "record"  ]
    }
  \end{dot2tex}
  \caption{\small\it La hash è un intero (senza segno) a 32 bit formato
    dallo XOR di queste tre righe.}
  \label{fig:hash}
\end{figure}

Le code hanno lo scopo di dividere i pacchetti tra i vari thread
\code{Updater} (uno per coda) al fine di garantire l'assenza di
interferenze al momento dell'aggiornamento della tabella hash. In
altre parole un thread dovrà poter accedere solamente ai propri slot
senza mai interferire con quelli dei suoi colleghi. Questo è garantito
dal modo in cui vengono smistati i pacchetti nelle code:

$$
indice\_della\_coda = hash \mbox{ \% } numero\_di\_threads
$$

Queste code sono dei buffer circolari che vengono usati senza
meccanismi di sincronizzazione, almeno dalla parte del \code{Capturer}
che si limita a segnalare la presenza di pacchetti; i thread
\code{Updater} aspettano quindi di essere segnalati per iniziare il
loro ciclo.

Se il \code{Capturer} dopo aver scelto una coda, trova che questa è
piena, scarta il pacchetto; per il motivo sopra descritto non può
delegarlo a un'altra coda. Per evitare allocazioni di memoria non
necessarie gli elementi delle code hanno un flag di uso.

Tutto questo può funzionare anche senza meccanismi di sincronizzazione
perché su ogni coda agiscono esattamente due thread: il primo
inserisce (e pone il flag a `usato') solo se un elemento non è usato;
il secondo rimuove (e pone il flag a `non usato') solo se un elemento
è usato. \`E importante notare che il flag viene modificato solo dopo
aver effettuato l'azione.

\subsection{Updater}
Dopo aver estratto un pacchetto dalla propria coda, un thread
\code{Updater} lo inserisce nella tabella hash; la posizione gli viene
passata dal \code{Capturer} che, come già detto, ha utilizzato per
scegliere la coda.

La tabella hash è implementata con liste di trabocco dove ogni
elemento ha, tra le altre cose, un flag di uso; questo approccio è
stato usato per limitare le allocazioni/deallocazioni dato che i
flussi devono essere memorizzati solo per un tempo limitato, dopo il
quale altri prenderanno il loro posto. Un altro motivo di questa
scelta riguarda l'interferenza con il thread \code{Flow~Dumper}: in
questo modo è possibile fare a meno di meccanismi di sincronizzazione,
seppur con qualche problema (vedere~§~\ref{par:issues}).

L'esatta procedura di aggiornamento della tabella hash all'arrivo di
un pacchetto (una volta individuata la lista di trabocco) è la
seguente:

\begin{enumerate}
\item la lista viene scorsa fino a che non si trova l'elemento da
  aggiornare o fino alla fine della lista; nel mentre viene salvato
  l'eventuale primo elemento non utilizzato;
\item se l'elemento viene trovato allora viene controllato il suo flag
  di uso: se è attivo viene aggiornato, altrimenti resettato; se
  invece viene raggiunta la fine della lista (quindi il pacchetto
  richiede un nuovo flusso) le opzioni sono due: se nella ricerca è
  stato trovato un elemento non usato allora viene utilizzato,
  altrimenti un nuovo nodo viene allocato alla fine della lista.
\end{enumerate}

\subsection{Flow Dumper}
Questo thread scorre ripetutamente la tabella hash in cerca di flussi
che sono scaduti, quando ne trova uno lo stampa (o lo passa al server
\code{HTTP} se è attivo) e commuta il flag in modo che il nodo possa
essere riutilizzato in futuro.

\subsection{Altri thread}
Oltre a questi c'è un altro thread che si occupa della gestione dei
segnali. Per aggiornare il database round-robin viene registrato un
timer (\code{setitimer}) che a intervalli regolari invia un segnale
(\code{SIGALRM}) al gestore, il quale tramite il comando
\code{RRDtool} inserisce i valori nel database.

\section{Problemi} \label{par:issues}
Nel ridurre a zero le operazioni di sincronizzazione (lock, wait,
etc.) sono occorsi alcuni problemi che tuttavia non compromettono la
stabilità del programma:

\begin{enumerate}
\item non deallocare mai i nodi inutilizzati della tabella hash può
  portare a lungo andare a un rallentamento globale causato
  dall'aumento della lunghezza media delle liste di trabocco, tuttavia
  problemi di memoria non dovrebbero verificarsi perché in caso di
  impossibilità di allocare un nodo, il pacchetto viene semplicemente
  scartato; e comunque dato il concetto di flusso, nuovi nodi si
  libereranno in continuazione.
\item un thread \code{Updater} prima di aggiornare un elemento
  controlla lo stato del flag di uso, se tra queste due operazioni il
  \code{Flow~Dumper} trova che il medesimo elemento è scaduto procede
  con il dump; il problema è che l'\code{Updater} lo aggiornerà
  comunque prolungando la durata del flusso, questo può portare nel
  migliore dei casi a due dump (entrambi corretti) riguardanti lo
  stesso flusso, mentre nel peggiore le informazioni che il
  \code{Flow~Dumper} leggerà saranno state in parte aggiornate
  dall'\code{Updater} con lo scarto di un pacchetto.
\end{enumerate}

\section{Utilizzo}
Eseguito senza parametri (per le opzioni vedere figura~\ref{fig:usage}),
\code{my-pcap} non mostra alcun output, ma tenta comunque di
effettuare la cattura.

\code{my-pcap} può catturare da una qualsiasi delle interfacce di
rete, enumerabili tramite l'opzione \code{l}. Una volta scelte le
opzioni cattura (\code{ipnd}) occorre scegliere una o più opzioni di
output (\code{rofw}, vedere~§~\ref{par:features}).

Per poter utilizzare il database round-robin è necessario prima
crearne uno con l'opzione \code{c}, è importante notare che per
effettuare la cattura è necessario essere super-utente, mentre il
database dovrebbe essere creato da un utente normale dato che una
volta iniziata la cattura i privilegi di \code{root} vengono
abbandonati, creando dei problemi se il programma è stato eseguito
tramite il comando \code{sudo} (vedere~§~\ref{par:notes}).

L'interfaccia \code{HTTP} (\code{w}) per avere una qualche utilità
deve essere usata insieme a una o entrambe le opzioni che abilitano la
gestione dei flussi e delle statistiche.

\begin{figure}
  \centering
\begin{verbatim}
usage:
    my-pcap -h
    my-pcap -c <rrd>
    my-pcap -g <rrd>
    my-pcap [-i <int>] [-p] [-n <max>] [-d i | o]
            [-r <rrd>] [-o] [-f] [-w <port>]

 -h        : print these infos.
 -l        : list compatible devices.
 -c <out>  : create a rrd database in <out>.
 -g <rrd>  : create a sample graph of <rrd>.
 -i <int>  : read packets from <int>; 'any' by default.
 -p        : promiscuous mode.
 -n <max>  : read <max> packets and exit.
 -d i|o    : capture direction, input(I) or output(O);
             input and output by default.
 -r <out>  : fill a rrd database in <out>.
 -o        : dump some info on stdout.
 -f        : dump some net flow info on stdout or http.
 -w <port> : enable http server on port <port>.

Analyze TCP and UDP over IPv4 packages only.
Use SIGTERM or SIGINT to break the process.
\end{verbatim}
  \caption{\small\it Risultato del comando `\code{my-pcap~-h}'.}
  \label{fig:usage}
\end{figure}

\section{Note} \label{par:notes}
Seguono alcune considerazioni sul progetto:

\begin{itemize}
\item La scelta di non deallocare mai i nodi inutilizzati porta
  inevitabilmente, come già detto, a un calo delle prestazioni; al
  momento non è certo se convenga evitare allocazioni/deallocazioni di
  memoria oppure mantenere le liste di trabocco il più corte
  possibile. Possibili soluzioni includono:
  \begin{itemize}
    \item l'uso di un mutex che garantisca la mutua esclusione tra un
      \code{Updater} e il \code{Flow~Dumper};
    \item accorgimenti nel trattare gli elementi della lista al fine
      di non causare interferenza tra i thread, anche mediante
      l'eliminazione ritardata dei nodi.
  \end{itemize}
\item Per una questione di semplicità \code{my-pcap} supporta
  solamente \code{IPv4}, anche se il passaggio alla versione
  successiva non dovrebbe causare troppi problemi.
\item La scelta di memorizzare i flussi in modo monodirezionale è
  dovuta al fatto che in questo modo la gestione dei flussi è più
  semplice.
\item Le statistiche dividono i pacchetti solamente in \code{TCP} e
  \code{UDP}, sarebbe più interessante andare avanti e monitorare
  l'utilizzo dei servizi (\code{HTTP}, \code{DNS}, \code{FTP}, etc.);
  per far questo basta analizzare il payload del livello applicazione
  cercando di riconoscere un pattern e/o in base alle porte
  utilizzate.
\item \code{my-pcap} fa uso di una quantità di parametri definiti nel
  codice tramite \code{\#define} che ne regolano profondamente il
  funzionamento; sono inizializzati a valori che non sono
  necessariamente la scelta migliore ma piuttosto la scelta più comoda
  per testare il programma.
\item La funzione hash (figura~\ref{fig:hash}) fa il suo lavoro ma è
  probabile che con una funzione differente si ottengano risultati
  migliori.
\item Come già accennato per catturare da \code{libpcap} occorre
  essere super-utente, quindi le possibilità sono due:
  \begin{enumerate}
  \item essere \code{root} e quindi ogni file prodotto ha i medesimi
    privilegi;
  \item eseguire \code{my-pcap} tramite il comando \code{sudo}, in
    questo caso il programma se ne accorge (controllando la variabile
    d'ambiente \code{SUDO\_USER}) e dopo aver richiesto il descrittore
    per la cattura ed eventualmente attivato l'interfaccia \code{HTTP}
    ripristina l'utente originale.
  \end{enumerate}
  Questa può forse non essere la scelta migliore in un contesto
  realistico, dove la cattura e l'analisi dei pacchetti spetterebbe ad
  un utente fittizio ad hoc, ma è senz'altro comoda in queste
  circostanze.
\item L'utilità pratica dell'interfaccia \code{HTTP} per come è adesso
  è decisamente scarsa, serve solo a mostrare un altro modo, oltre la
  stampa su \code{stdout}, di esportare i dati ottenuti durante la
  cattura.
\item Sempre a riguardo dell'interfaccia \code{HTTP}, sarebbe utile
  stabilire su quale indirizzo far partire il servizio, se non altro
  per evitare di catturare il traffico di chi accede all'interfaccia;
  questo è impossibile con la versione di \code{libmicrohttpd} usata,
  purtroppo quella più recente dà attualmente qualche problema.
\end{itemize}

\end{document}
