\section{Commenti finali}
\subsection{Quindi, cosa bisogna aspettarsi dal monitoraggio di rete?}
\begin{itemize}
\item Capacit� di individuare in maniera automatica quei problemi che sono costantemente sotto monitoraggio (ad esempio, non c'� traffico su di un link della backbone: la rete � caduta?).
\item Ricevere allarmi a proposito di potenziali (ad esempio l'utilizzo della CPU � troppo alto) e reali (ad esempio li disco � pieno) problemi.
\item Notifica e ripristino automatico di problemi noti con note soluzioni (ad esempio, se il link dell'email non va viene usato un link di backup).
\item Notificare all'uomo tutti quei problemi che necessitano attenzione e che non possono essere ripristinati (ad esempio l'host X non � raggiungibile).
\end{itemize}

\subsection{Avvertenze sul monitoraggio}
\begin{itemize}
\item Se un'applicazione necessita di assistenza umana per quei problemi che possono essere risolti in maniera automatica, allora l'uso di questa applicazione non � completamente vantaggioso.
\item Gli allarmi (sicurezza al 100\% che c'� qualcosa che non va) sono diversi dagli avvisi (potrebbe esserci qualche problema): non si pretenda di essere precisi/catastrofici se non � il caso.
\item Gli allarmi sono inutili se non c'� nessuno che li controlla.
\item Troppi (falsi) allarmi equivale a non avere allarmi: gli umani tendono ad ignorare i fatti quando qualcuno di essi � falso.
\end{itemize}

% LocalWords:  backbone l'host
